\documentclass[aspectratio=169]{beamer}
% \documentclass[aspectratio=169, handout]{beamer}

\input{preamble.tex}

\title{Future Climate}
\author{Timescales}

\begin{document}

% (Branson) Friday March 10th - Lecture 45 – Future Climate 2: Timescales. 
% Introduce the idea that the climate system has momentum and inertia.
% What changes are we already committed to, and what are their timescales?
% What does recovery look like?

\begin{frame}{Future Climate\\Timescales \& Recovery}
    \includegraphics[width=\linewidth, totalheight=0.8\textheight, keepaspectratio]{future-palaeo-sensitivity.jpeg}
\end{frame}

\begin{frame}{Aims}

    \begin{enumerate}
        \item There is still \textbf{considerable uncertainty} in climate models.
        \item \textbf{Climate Sensitivity} is a crucial unknown.
        \item Uncetainties arise from pushing the climate into an \textbf{unseen state}.
        \item \textbf{Past Climates} offer constraints on sensitivity\dots
        \item \dots and insights into \textbf{recovery}.
        \item \textbf{Climate Action is Urgent!}
    \end{enumerate}

\end{frame}

\section{(Un)certainty}

\begin{frame}{Uncertainty}
    % weather vs. climate - specific predictions hard; average predictions more certain
    \centering
    \includegraphics[width=\linewidth, totalheight=0.82\textheight, keepaspectratio]{future-Tabaldi-CMIP6-TAS.png}
    \slidereference{Tabaldi et al., 2021}
\end{frame}

\begin{frame}<handout:0>{Uncertainty}
    % weather vs. climate - specific predictions hard; average predictions more certain
    \centering
    \includegraphics[width=\linewidth, totalheight=0.82\textheight, keepaspectratio]{future-Lehner-uncertainty.png}
    \slidereference{Lehner et al., 2020}
\end{frame}

\begin{frame}{Uncertainty: Scenario}
    % weather vs. climate - specific predictions hard; average predictions more certain
    \centering
    \includegraphics[width=\linewidth, totalheight=0.82\textheight, keepaspectratio]{future-emission-scenarios.png}
    \slidereference{IPCC AR6, SPM}
\end{frame}

\begin{frame}{Uncertainty}
    % weather vs. climate - specific predictions hard; average predictions more certain
    \centering
    \includegraphics[width=\linewidth, totalheight=0.82\textheight, keepaspectratio]{future-Lehner-uncertainty.png}

    \onslide<2|handout:1>{$\sim$ 40-80 \% of uncertainty comes from model choice!}

    \slidereference{Lehner et al., 2020}
\end{frame}

\section{Future Climate}

\begin{frame}{Certainty}
    \centering
    \includegraphics[width=\linewidth, totalheight=0.82\textheight, keepaspectratio]{future-SPM_Figure_10.png}
    \slidereference{IPCC AR6, SPM}
\end{frame}

\begin{frame}{Certainty: Feedbacks}
    \centering
    \includegraphics[width=\linewidth, totalheight=0.82\textheight, keepaspectratio]{future-SPM_Figure_7.png}
    \slidereference{IPCC AR6, SPM}
\end{frame}

\begin{frame}{Certainty: Commitments}
    \centering
    \includegraphics[width=\linewidth, totalheight=0.75\textheight, keepaspectratio]{future-TCR-ECS-AR3.png}
    \slidereference{IPCC AR3}
\end{frame}

\section{Climate Sensitivity}

\begin{frame}{Climate Sensitivity?}
    \centering
    \includegraphics[width=\linewidth, totalheight=0.75\textheight, keepaspectratio]{future-AR6_WGI_Figure_7_18.0.png}
    \slidereference{IPCC AR6}
\end{frame}


% https://www.carbonbrief.org/explainer-how-scientists-estimate-climate-sensitivity/

% \begin{frame}{Climate Sensitivity}
%     \centering
%     \includegraphics[width=\linewidth, totalheight=0.75\textheight, keepaspectratio]{future-model-sensitivity.png}
%     \slidereference{IPCC AR6}
% \end{frame}

\begin{frame}{Can we use models to estimate sensitivity?}
    \centering
    \includegraphics[width=\linewidth, totalheight=0.75\textheight, keepaspectratio]{future-model-hindcast.png}

    % But IPCC models are 'ground truthed' against the modern world... (narrow range of conditions, hindcast figure)    
    % Complex feedbacks (Alex's Fig)

\end{frame}

\begin{frame}{Transient Climate Response}
    \centering
    \includegraphics[width=\linewidth, totalheight=0.75\textheight, keepaspectratio]{future-AR6_WGI_Figure_7_18.1.png}
    \slidereference{IPCC AR6}
\end{frame}

\begin{frame}{Transient Climate Response}
    \centering
    \includegraphics[width=\linewidth, totalheight=0.82\textheight, keepaspectratio]{future-SPM_Figure_10.png}
    \slidereference{IPCC AR6, SPM}
\end{frame}

\begin{frame}{Equilibrium Climate Sensitivity?}
    \centering
    \includegraphics[width=\linewidth, totalheight=0.75\textheight, keepaspectratio]{future-TCR-ECS-AR3.png}
    \slidereference{IPCC AR3}
\end{frame}

\begin{frame}{Feedback Timescales}
    \centering
    \includegraphics[width=\linewidth, totalheight=0.75\textheight, keepaspectratio]{future-feedback-timescales.jpg}
    \slidereference{Knutti \& Rugenstein, 2015}
\end{frame}

\begin{frame}{Equilibrium Climate Sensitivity}
    \centering
    \includegraphics[width=\linewidth, totalheight=0.75\textheight, keepaspectratio]{future-AR6_WGI_Figure_7_18.2.png}
    Larger uncertainty in ECS because the change is \textbf{unprecedented.}
    \slidereference{IPCC AR6}
\end{frame}

\begin{frame}{Precedent?}
    % we are pushing the climate into an unknown regime.
    \centering
    \includegraphics[width=\linewidth, totalheight=0.8\textheight, keepaspectratio]{future-palaeo-co2.jpeg}
\end{frame}            

\section{Past Climate}

\begin{frame}{Climate Records}
    \centering
    \includegraphics<1|handout:1>[width=\linewidth, totalheight=0.8\textheight, keepaspectratio]{future-IODP-ship.png}
    \includegraphics<2|handout:2>[width=\linewidth, totalheight=0.8\textheight, keepaspectratio]{future-iodp_platforms.jpg}
    \includegraphics<3|handout:3>[width=\linewidth, totalheight=0.8\textheight, keepaspectratio]{future-iodp-cores.png}
    \includegraphics<4|handout:4>[width=\linewidth, totalheight=0.8\textheight, keepaspectratio]{future-iodp_odp_dsdp.jpg}
\end{frame}

\begin{frame}{Palaeo-\ce{pCO2}}
    % how do we know what happened in past climates?
    % palaeo-CO2
    \centering
    \includegraphics<1>[width=\linewidth, totalheight=0.8\textheight, keepaspectratio]{future-palaeo-co2-records.png}
    \includegraphics<2|handout:0>[width=\linewidth, totalheight=0.8\textheight, keepaspectratio]{future-palaeo-co2-records.1.png}
\end{frame}

\begin{frame}{Palaeo-\ce{pCO2}}
    % how do we know what happened in past climates?
    % palaeo-CO2
    \centering
    \includegraphics[width=\linewidth, totalheight=0.8\textheight, keepaspectratio]{future-boron-proxies.png}
\end{frame}

\begin{frame}{Palaeo-\ce{pCO2}}
    % how do we know what happened in past climates?
    % palaeo-CO2
    \centering
    \includegraphics[width=\linewidth, totalheight=0.8\textheight, keepaspectratio]{future-IPCC_AR6_WGI_Figure_5_3.png}
    \slidereference{IPCC AR6}
\end{frame}

\begin{frame}{Palaeo-Temperature}
    % palaeo-temperature
    \centering
    \includegraphics[width=\linewidth, totalheight=0.8\textheight, keepaspectratio]{future-palaeo-temperature.jpeg}
    \slidereference{Westerhold et al. (2020); CENOGRID}
\end{frame}

\begin{frame}{Palaeo-Temperature}
    % palaeo-temperature
    \centering
    \includegraphics[width=\linewidth, totalheight=0.8\textheight, keepaspectratio]{future-CENOGRID-temperature.PNG}
    \slidereference{CENOGRID}
\end{frame}

\begin{frame}{Palaeo-Sensitivity}
    \begin{columns}
        \begin{column}{0.5\linewidth}
            \includegraphics[width=\linewidth, totalheight=0.8\textheight, keepaspectratio]{future-palaeo-sensitivity.jpeg}
        \end{column}
        \onslide<2>{
            \begin{column}{0.5\linewidth}
                \includegraphics[width=\linewidth, totalheight=0.8\textheight, keepaspectratio]{future-AR6_WGI_Figure_7_18.3.png}
            \end{column}
        }
    \end{columns}
\end{frame}

\begin{frame}{Equilibrium Climate Sensitivity}
    \centering
    \includegraphics[width=\linewidth, totalheight=0.75\textheight, keepaspectratio]{future-AR6_WGI_Figure_7_18.png}
    \slidereference{IPCC AR6}

    \onslide<2|handout:1>{If we stop emitting \textbf{now}, we are committed additional temperature rise \\ (1.5-4x current warming).}
    % even if we stop emitting now, temperatures will rise for centuries, between 1.5-4x current warming.
\end{frame}

\section{Recovery}

\begin{frame}{Timescales of Recovery: PETM}
    \centering
    \includegraphics[width=\linewidth, totalheight=0.8\textheight, keepaspectratio]{future-palaeo-co2.1.jpeg}
\end{frame}

\begin{frame}{Timescales of Recovery: PETM}
    \centering
    \includegraphics[width=\linewidth, totalheight=0.8\textheight, keepaspectratio]{future-palaeo-temperature.jpeg}
\end{frame}


\begin{frame}{PETM: Mechanism?}
    \centering
    \includegraphics[width=\linewidth, totalheight=0.8\textheight, keepaspectratio]{future-petm-mechanism.1.png}
    \slidereference{Glikson (2014)}
\end{frame}

\begin{frame}{PETM: Consequences}
    \centering
    \includegraphics[width=\linewidth, totalheight=0.7\textheight, keepaspectratio]{future-ccd-mechanism.png}
    \slidereference{Middelburg (2018)}
\end{frame}

\begin{frame}{PETM: Consequences}
    \centering
    \includegraphics[width=\linewidth, totalheight=0.8\textheight, keepaspectratio]{future-petm-core.jpg}
    \slidereference{ECORD}
\end{frame}

\begin{frame}{PETM: Recovery}
    \centering
    \includegraphics[width=\linewidth, totalheight=0.8\textheight, keepaspectratio]{future-petm-recovery.jpeg}
    \slidereference{Zachos et al. (2005)}
\end{frame}

\begin{frame}{PETM: Recovery}
    \centering
    \includegraphics[width=\linewidth, totalheight=0.7\textheight, keepaspectratio]{future-petm-recovery-stack.jpeg}
    \slidereference{Zachos et al. (2005)}
\end{frame}

\begin{frame}{PETM: Comparison?}
    \centering
    \includegraphics[width=\linewidth, totalheight=0.7\textheight, keepaspectratio]{future-petm-modern.png}
    \slidereference{Zeebe (2012)}
\end{frame}

\section{Urgency}

\begin{frame}{Urgent Action!}
    \centering
    \includegraphics[width=\linewidth, totalheight=0.7\textheight, keepaspectratio]{future-IPCC_AR6_WGI_TS_Figure_18.png}
    \slidereference{IPCC AR6}
    \textbf{Uncertainty should not lead to inactivity!}\\Estimates based on Transient Climate Response alone are pretty dire\dots
\end{frame}

\end{document}





% \begin{frame}{TITLE}
% \end{frame}